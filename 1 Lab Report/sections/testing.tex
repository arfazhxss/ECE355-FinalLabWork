\section{Testing and Results}
\subsection{Testing Procedure}
To validate the functionality and performance of the system, a structured testing procedure was employed. Each aspect of the system was rigorously examined to ensure reliability and accuracy under real-world conditions.

\subsubsection{System Initialization and Peripheral Verification}
\begin{enumerate}[leftmargin=2em]
    \item Power on the STM32F0 Discovery board and ensure stable voltage levels across all peripherals.
    \item Verify system clock configuration by outputting the clock frequency to the console and ensuring it matches the expected 48 MHz.
    \item Test GPIO pin configurations by probing each pin and confirming the expected input/output states using an oscilloscope.
    \item Check ADC and DAC initialization by generating a test analog signal using a function generator and confirming correct ADC reads and DAC outputs using a multimeter.
    \item Initialize the OLED display and verify proper communication via SPI by displaying test data.
\end{enumerate}

\subsubsection{Frequency Measurement Validation}
\begin{enumerate}[leftmargin=2em]
    \item Generate a square-wave signal (0–3.3 V amplitude) using a function generator at various frequencies (10 Hz to 1 MHz).
    \item Connect the signal to PA2 and monitor the EXTI2 interrupt behavior using the debugger or console outputs.
    \item Measure the time elapsed between two rising edges of the input signal using TIM2 and confirm accuracy by comparing the calculated frequency to the function generator's set frequency.
    \item Determine the minimum detectable frequency by lowering the input signal frequency until the system fails to provide consistent measurements.
    \item Determine the maximum detectable frequency by increasing the input signal frequency until the timer fails to capture events accurately.
\end{enumerate}

\subsubsection{Resistance Calculation Validation}
\begin{enumerate}[leftmargin=2em]
    \item Connect a potentiometer to PA5 and adjust its resistance across the full range (0–5 k$\omega$).
    \item Confirm ADC readings at various potentiometer positions by comparing the measured resistance (via the ADC-to-resistance formula) to values obtained using a multimeter.
    \item Validate real-time updates on the OLED display for resistance measurements.
\end{enumerate}

\subsubsection{Mode Switching Functionality}
\begin{enumerate}[leftmargin=2em]
    \item Test the system's response to pressing the USER button (PA0) by toggling between NE555 Timer Mode and Function Generator Mode.
    \item Ensure proper reconfiguration of EXTI interrupts for PA1 and PA2 when switching modes.
    \item Validate that frequency measurements from the correct input source are displayed on the OLED after each mode switch.
    \item Confirm seamless transitions without data loss or system crashes.
\end{enumerate}

\subsubsection{Signal Monitoring and DAC Control}
\begin{enumerate}[leftmargin=2em]
    \item Adjust the potentiometer and monitor the real-time ADC readings.
    \item Validate DAC outputs by connecting an oscilloscope to PA4 and confirming the output voltage corresponds to the scaled ADC input.
    \item Ensure that the DAC signal drives the optocoupler to adjust the PWM frequency and duty cycle of the NE555 timer circuit.
    \item Verify continuous synchronization between ADC input, DAC output, and OLED display updates.
\end{enumerate}

\subsubsection{Noise and Stability Testing}
\begin{enumerate}[leftmargin=2em]
    \item Introduce electrical noise to the system using a signal generator or by varying the power supply voltage.
    \item Verify the robustness of the ADC readings and DAC outputs under noisy conditions by observing the stability of displayed data and oscilloscope waveforms.
    \item Test the impact of simultaneous mode switching and signal input variations on system performance.
    \item Confirm the effectiveness of decoupling capacitors in mitigating noise.
\end{enumerate}

\subsection{Results}
The system demonstrated robust performance across all testing scenarios, meeting or exceeding the project requirements.

\subsubsection{Frequency Measurement Performance}
\begin{itemize}[leftmargin=2em]
    \item \textbf{Accuracy:} Measured frequencies deviated by less than 2\% from the function generator's reference values across a range of 10 Hz to 500 kHz.
    \item \textbf{Minimum Detectable Frequency:} 10 Hz, limited by the timer's ability to measure long periods without overflow.
    \item \textbf{Maximum Detectable Frequency:} 500 kHz, constrained by EXTI interrupt latency and TIM2 resolution.
\end{itemize}

\subsubsection{Resistance Measurement Performance}
\begin{itemize}[leftmargin=2em]
    \item \textbf{Accuracy:} Resistance calculations matched multimeter readings within 5\% across the full range of the potentiometer (0–5 k$\omega$).
    \item \textbf{Response Time:} Real-time updates to the OLED display occurred with minimal latency, ensuring user-friendly interaction.
\end{itemize}

\subsubsection{Mode Switching Reliability}
\begin{itemize}[leftmargin=2em]
    \item \textbf{Seamless Transitions:} Mode switches occurred instantaneously, with accurate updates to frequency and resistance measurements.
    \item \textbf{Interrupt Handling:} No missed events or system crashes were observed during rapid mode toggling.
\end{itemize}

\subsubsection{Noise Mitigation and Stability}
\begin{itemize}[leftmargin=2em]
    \item \textbf{Noise Rejection:} Effective decoupling capacitors and robust interrupt-driven control minimized noise-induced errors.
    \item \textbf{Stable Operation:} The system maintained consistent performance even under noisy conditions and power supply variations.
\end{itemize}

\subsubsection{System Limitations}
\begin{itemize}[leftmargin=2em]
    \item The maximum detectable frequency (500 kHz) is limited by EXTI and TIM2 latency, and further optimization may require higher-performance hardware.
    \item The ADC sampling rate restricts the system's ability to process rapidly changing analog inputs.
\end{itemize}
