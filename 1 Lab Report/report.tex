\documentclass[12pt]{article}
\usepackage{pgfplots}
\pgfplotsset{compat=1.18}

% Import essential packages from imports.tex
% Essential packages
\usepackage[utf8]{inputenc}
\usepackage{graphicx}
\usepackage{amsmath}
\usepackage{listings}
\usepackage{hyperref}
\usepackage{color}
\usepackage{float}
\usepackage[margin=1in]{geometry}
\usepackage{enumitem}
\usepackage{titlesec}
\usepackage{fancyhdr}
\usepackage{xcolor} % For custom colors
\usepackage{listings} % For styled code listings
\usepackage[font=small,labelfont=bf]{caption}

% References
\usepackage[backend=bibtex,style=ieee]{biblatex}
\addbibresource{sections/report.bib}
\defbibheading{bibliography}{}

% Custom color definitions
\definecolor{mygray}{rgb}{0.5,0.5,0.5}
\definecolor{mymauve}{rgb}{0.58,0,0.82}
\definecolor{darkblue}{rgb}{0,0,0.55}
\definecolor{violet}{rgb}{0.5,0,0.5}

% Custom settings for lstlisting
\lstset{
  language=C,
  basicstyle=\footnotesize\ttfamily,
  numbers=left,
  numberstyle=\tiny\color{mygray},
  stepnumber=1,
  numbersep=5pt,
  backgroundcolor=\color{white},
  showspaces=false,
  showstringspaces=false,
  showtabs=false,
  frame=single,
  rulecolor=\color{black},
  tabsize=4,
  captionpos=b,
  breaklines=true,
  breakatwhitespace=false,
  keywordstyle=[1]\bfseries\color{darkblue},
  keywordstyle=[2]\bfseries\color{violet},
  commentstyle=\color{mygray},
  stringstyle=\color{mymauve},
  escapeinside={\%*}{*)},
  morekeywords={*,...}
}

% Header and footer settings
\pagestyle{fancy}
\fancyhead{}
\fancyfoot{}
\fancyfoot[C]{\thepage}
\renewcommand{\headrulewidth}{0.4pt}


\begin{document}

% Title page
\begin{titlepage}
\begin{center}

% University and Department
{\Large\bfseries\scshape University of Victoria \par}
{\large Department of Electrical and Computer Engineering \par}
\vspace{1.5cm}

% Report Title
{\Huge\bfseries Lab Project Report \par}
\vspace{0.4cm}
{\LARGE\textit{Function Generator and ADC/DAC System Implementation} \par}
\vspace{1.2cm}

% Submission Details
{\large\bfseries Submitted By: \par}
\vspace{0.4cm}
\begin{tabular}{rl}
Group Number: & 04 \\
Section: & B0X \\
Members: & Arfaz Hossain (V00984826) \\
         & Aly Mooltazeem (V00962689) \\
\end{tabular}
\vspace{1cm}

% Faculty Information
{\large\bfseries Faculty: \par}
\vspace{0.4cm}
\begin{tabular}{rl}
Lecture Professor: & Daler Rakhmatov \\
Lab Technologist: & Brent Sirna \\
\end{tabular}
\vspace{1cm}

% Date of Submission
{\large\bfseries Date of Submission: Nov 29, 2024 \par}
\vspace{1.5cm}

% Marking Table
\begin{tabular}{|p{8cm}|c|}
\hline
\textbf{Section} & \textbf{Marks} \\
\hline
Problem Description/Specifications & 5 \\
Design/Solution & 15 \\
Testing/Results & 10 \\
Discussion & 15 \\
Code Design and Documentation & 15 \\
\hline
\textbf{Total} & 60 \\
\hline
\end{tabular}

\end{center}
\end{titlepage}


\tableofcontents
\newpage

% Abstract Section
\section*{Abstract}
\addcontentsline{toc}{section}{Abstract}
This project focuses on the design and implementation of measuring the frequencies of the PWM signals inputted into the microcontroller. Key objectives include signal measurement via an ADC, signal generation through a DAC, computation of frequency and resistance using mathematical models, and real-time visualization on an OLED display. The system seamlessly integrates peripherals such as GPIO, TIM2, ADC, DAC, and EXTI, enabling efficient operation across two modes—"Function Generator" and "ADC/DAC"—toggled via a user button. Rigorous testing demonstrated high accuracy in frequency and resistance calculations, with deviations limited to less than 2\% under controlled conditions. Challenges, including signal noise and interrupt conflicts, were effectively mitigated through synchronization and advanced filtering techniques. This work highlights modularity, efficient peripheral management, and avenues for future improvements, such as enhanced sampling rates and advanced noise suppression algorithms.

% Problem Description and Specifications
\section{Problem Description and Specifications}
\subsection{Objectives}
The primary goal is to measure the frequency of the square-wave signal from the function generator and the PWM signal the NE555 timer through ADC/DAC system featuring:
\begin{itemize}[leftmargin=2em]
    \item \textbf{Signal Measurement:} ADC-based real-time input signal processing.
    \item \textbf{Signal Generation:} DAC output for test signals.
    \item \textbf{Computations:} Frequency and resistance calculations using mathematical models.
    \item \textbf{Visualization:} Display of results on an OLED for user feedback.
\end{itemize}

\subsection{Specifications}
The system specifications are as follows:
\begin{itemize}[leftmargin=2em]
    \item \textbf{12-bit ADC:} Captures accurate signal data.
    \item \textbf{DAC Output:} Generates analog signals based on processed input (0 to 3.3 volts).
    \item \textbf{Computation Models:} 
    \begin{itemize}
        \item Frequency: Derived from signal timing data.
        \item Resistance: Calculated using voltage divider equations.
    \end{itemize}
    \item \textbf{OLED Display:} Real-time visualization for analysis.
    \item \textbf{Mode Switching:} Toggle between Function Generator and ADC/DAC modes.
\end{itemize}

\newpage

% Design and Solution
\section{Design and Solution}
\subsection{System Overview}
The STM32F051R8 microcontroller is the central processor interfacing with:
\begin{itemize}[leftmargin=2em]
    \item \textbf{ADC:} Captures signals at 12-bit resolution.
    \item \textbf{DAC:} Outputs processed analog signals.
%    \item \textbf{TIM2:} Measures signal timing for frequency computations.
    \item \textbf{GPIO:} Handles external button inputs for mode switching.
%    \item \textbf{OLED Display:} Provides real-time feedback on computed results.
\end{itemize}

\subsection{Hardware Design}
\subsubsection{Block Diagram}
%\begin{center}
%    \includegraphics[width=0.8\textwidth]{path_to_block_diagram.png} % Include your block diagram
%\end{center}

\subsubsection{Key Components and Connections}
\begin{itemize}[leftmargin=2em]
    \item \textbf{ADC Input:} Configured on pin PA5 for analog signal capture.
    \item \textbf{DAC Output:} Configured on pin PA4 for real-time output.
    \item \textbf{Mode-Switch Button:} External interrupt on pin PA0.
    \item \textbf{OLED Display:} SPI communication for real-time data visualization.
    \item \textbf{Power Supply:} Regulated 3.3V for stable operation.
\end{itemize}

\subsection{Software Design}
The software architecture includes:
\begin{itemize}[leftmargin=2em]
    \item \textbf{Initialization Functions:} Setting up system clocks and peripherals.
    \item \textbf{Interrupt Handlers:} Managing button presses for mode toggling (and others to be included).
    \item \textbf{Computation Algorithms:} Frequency and resistance calculations.
\end{itemize}

\newpage

\subsubsection{Code Snippet: System Clock Initialization}
\begin{lstlisting}[caption=System Clock Initialization, label=lst:SystemClockInit]
void SystemClock48MHz(void) {
    RCC->CR &= ~(RCC_CR_PLLON);  // Disable PLL
    while ((RCC->CR & RCC_CR_PLLRDY) != 0);  // Wait for unlock
    RCC->CFGR = 0x00280000;  // Configure PLL
    RCC->CR |= RCC_CR_PLLON;  // Enable PLL
    while ((RCC->CR & RCC_CR_PLLRDY) != RCC_CR_PLLRDY);  // Lock PLL
    RCC->CFGR = (RCC->CFGR & (~RCC_CFGR_SW_Msk)) | RCC_CFGR_SW_PLL; // Switch clock
    SystemCoreClockUpdate();
}
\end{lstlisting}

% Testing and Results
\section{Testing and Results}
\subsection{Testing Procedure}
\begin{enumerate}[leftmargin=2em]
    \item Initialize the system and ensure peripheral communication.
    \item Measure and compute frequency and resistance.
    \item Validate ADC and DAC outputs in real-time.
    \item Toggle modes and observe transitions.
\end{enumerate}

\subsection{Results}

\subsubsection{Frequency Measurements}
\label{subsubsec:results}

The graph below shows the plot of measured frequencies with the frequencies from the function generator.

\begin{figure}[htbp]
    \centering
    \begin{tikzpicture}
        \begin{axis}[
            xlabel={Frequency (Hz)},       % Label for the x-axis
            ylabel={Measured Frequency (Hz)},       % Label for the y-axis
            grid=both,               % Add a grid
        grid style={dotted},
            width=12cm,              % Adjust width
            height=8cm,              % Adjust height
            legend style={at={(0.5,-0.2)}, anchor=north, legend columns=-1},
            legend entries={Expected Curve, Actual Curve},
            legend cell align={left}
        ]
            \addplot[
                color=blue,         % Line color,
                dashed,
                mark=*              % Marker style
            ] 
            table[
                col sep=comma,      % Column separator (comma for CSV files)
                x=Input,                % Specify column for x
                y=Expected                 % Specify column for y
            ] {tables/freq_measurements.csv};

            \addplot[
                color=red,         % Line color,
                mark=*              % Marker style
            ] 
            table[
                col sep=comma,      % Column separator (comma for CSV files)
                x=Input,                % Specify column for x
                y=Measured                 % Specify column for y
            ] {tables/freq_measurements.csv};
       
        \end{axis}
    \end{tikzpicture}
    \caption{A plot of measured frequencies with the frequencies from the function generator.}
    \label{fig:frequencies}
\end{figure}

\begin{tikzpicture}
    \begin{axis}[
        xlabel={Frequency (Hz)},       % Label for the x-axis
        ylabel={Percentage Error (\%)},       % Label for the y-axis
        grid=both,               % Add a grid
    grid style={dotted},
        width=12cm,              % Adjust width
        height=8cm,              % Adjust height
        legend style={at={(0.5,-0.2)}, anchor=north, legend columns=-1},
        legend cell align={left}
    ]

        \addplot[
            color=cyan,         % Line color,
            mark=*              % Marker style
        ] 
        table[
            col sep=comma,      % Column separator (comma for CSV files)
            x=Input,                % Specify column for x
            y=Error                 % Specify column for y
        ] {tables/freq_measurements.csv};
   
    \end{axis}
\end{tikzpicture}


\begin{itemize}[leftmargin=2em]
    \item \textbf{Frequency Calculation Accuracy:} Deviation $<$ 2\%.
    \item \textbf{Resistance Calculation Accuracy:} Robust under varying loads.
    \item \textbf{Noise Mitigation:} Effective filtering techniques implemented.
\end{itemize}



% Discussion and Conclusion
\section{Discussion}
\subsection{Challenges and Limitations}

The system can only measure frequencies at certain ranges. It is observed from Section~\ref{subsubsec:results} that it can measure up to approximately 500,000 Hz. Higher frequencies have smaller time period, $T$. As mentioned in section [TBD], it counts the number of counts using the TIM2 counter between two rising edges using the prescaler value set at zero (i.e. at system clock frequency of 48 MHz).  However, the system clock lacks adequate sampling frequency to measure very small periods, which makes the frequency measurement inaccurate.

For a 48 MHz clock, each clock cycle is \textbf{20.83 ns}. The number of counts, \( N \), recorded by the timer for one signal period ($T$) is:

\begin{equation}\label{number_of_counts}
    N = T \times f_{\text{clock}}
\end{equation}

Where $T$ is the period of the signal (in seconds), and $f_{\text{clock}}$ is the system clock frequency (48 MHz).

For a high-frequency signal like 600 kHz, the period, $T$ of the signal is:
$$T = \frac{1}{f_{\text{signal}}} = \frac{1}{600,000} = 1.6667 \, \mu\text{s}$$

Therefore, the number of counts for this frequency are:
$$N = T \times f_{\text{clock}} = 1.6667 \times 10^{-6} \times 48 \times 10^6 = 80 \, \text{counts}$$

With only 80 counts, a missed or extra clock cycle introduces significant error. For example, a missing \textbf{1 clock cycle} changes $N$ to 79, resulting in:
    \[
    \text{Frequency error} = \frac{f_{\text{clock}}}{N} - \frac{f_{\text{clock}}}{N-1}
    \]
    \[
    \text{Error} = \frac{48,000,000}{80} - \frac{48,000,000}{79} \approx 600,000 - 607,595 = 7,595 \, \text{Hz}
    \]
This is a \textbf{quantization error} of approximately \textbf{1.27\%} just from missing 1 clock cycle. However, it missed many clock cycles which gave a relative error of about 49.36\%. It is very likely caused by the following criteria:

\begin{itemize}[leftmargin=2em]
    \item  \textbf{Interrupt Latency:} The microcontroller might not be able to process interrupts fast enough due to delays in handling and returning from interrupts. Therefore, the interrupt handling has caused some synchronization issues.
    \item \textbf{Jitter and Noise:} At such high frequencies, any signal noise or jitter can cause additional rising edges to be detected erroneously or edges to be missed altogether, introducing further inaccuracies.
\end{itemize}

The timers in the STM32 have an auto-reload register (\texttt{TIM2->ARR}). The ARR sets the maximum count value of the timer, effectively determining the timer's period. When the timer reaches the defined maximum value, it generates an interrupt. In this code, it resets the count value in the timer to zero. This is done to prevent any overflow issues.
\\[6pt]
In the STM32 microcontroller, the ARR is allocated to a finite size of 32 bits due to hardware limitations. The maximum count is determined by the following:

$$N_{\text{max}} = 2^{32} - 1 = 4,294,967,295$$

Therefore, the \textbf{minimum frequency} would be:
$$f_{\text{min}} = \frac{f_{\text{clock}}}{N_{\text{max}}} = \frac{48,000,000}{4,294,967,295} \approx 0.0112 \text{ Hz}$$

\subsection{Future Work}
\begin{itemize}[leftmargin=2em]
    \item \textbf{Integrate advanced noise reduction algorithms:} Implement digital filters (e.g., low-pass or notch filters) in the firmware to reduce jitter and signal noise in high-frequency measurements.

    \item \textbf{Consider microcontrollers with higher clock frequencies:} Migrate to a microcontroller with a faster system clock (e.g., 72 MHz or higher) to:
    \begin{itemize}[leftmargin=2em]
        \item Achieve finer timer granularity for shorter periods.
        \item Reduce the risk of missing high-frequency edges due to interrupt latency or slow execution.
        \item Expand the measurable frequency range without introducing inaccuracies.
    \end{itemize}

    \item \textbf{Utilize timer chaining for extended range:} Combine multiple timers (e.g. TIM2, TIM3, TIM16, etc.) to measure very low frequencies without encountering timer overflow, extending the detectable frequency range.

    \item \textbf{Introduce dynamic prescaling:} Implement an adaptive prescaler to automatically adjust timer resolution based on signal frequency, ensuring better accuracy across a wide range of frequencies.

    % \item Explore DMA integration.**
    % - Use Direct Memory Access (DMA) to offload data collection tasks from the CPU, enabling smoother and faster processing of high-frequency signals.

    \item \textbf{Enable multi-channel frequency detection:} Add support for simultaneous frequency measurement on multiple signal inputs using additional GPIO pins and timers.


\end{itemize}

\section{Conclusion}
The project demonstrates a reliable, modular design for real-time signal processing, providing a strong foundation for further enhancements.

% References
\section{References}
\begin{enumerate}[leftmargin=2em]
    \item STM32F051R8 Datasheet and Reference Manual.
    \item Academic resources on ADC/DAC systems.
    \item Industry best practices for signal processing.
\end{enumerate}

\end{document}