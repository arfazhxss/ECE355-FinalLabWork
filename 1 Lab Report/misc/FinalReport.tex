\documentclass{article}
\usepackage{amsmath}
\usepackage{graphicx}
\usepackage{listings}
\usepackage{tikz}
\usepackage{circuitikz}
\usepackage{algorithm}
\usepackage{algpseudocode}

\title{Comprehensive Analysis of STM32F0-Based Frequency Measurement System}
\author{Technical Documentation Team}
\date{\today}

\begin{document}

\maketitle

\tableofcontents

\section{System Overview}
\subsection{Architecture Specifications}

The system is built around the STM32F051x8 microcontroller, operating at 48MHz through PLL multiplication. The core architecture implements:

\begin{equation}
f_{SYSCLK} = f_{HSI} \times \text{PLL}_{\text{multiplier}} = 8\text{MHz} \times 6 = 48\text{MHz}
\end{equation}

\subsection{Core Components}

The system comprises the following fundamental blocks:

\begin{itemize}
\item \textbf{Input Processing Unit (IPU)}
    \begin{itemize}
    \item NE555 Timer interface (PA1)
    \item Function Generator interface (PA2)
    \item Mode Selection Button (PA0)
    \item Potentiometer ADC input (PA5)
    \end{itemize}
\item \textbf{Output Processing Unit (OPU)}
    \begin{itemize}
    \item DAC output channel (PA4)
    \item OLED display interface
    \end{itemize}
\item \textbf{Processing Core}
    \begin{itemize}
    \item Timer-based frequency measurement
    \item Real-time signal processing
    \item Mode switching logic
    \end{itemize}
\end{itemize}

\section{Hardware Implementation}
\subsection{GPIO Configuration Details}

The GPIO configuration follows the following register-level setup:

\begin{equation}
\text{MODER}_{\text{PA0}} = 00_2 \text{ (Input Mode)}
\end{equation}
\begin{equation}
\text{MODER}_{\text{PA1}} = 00_2 \text{ (Input Mode)}
\end{equation}
\begin{equation}
\text{MODER}_{\text{PA2}} = 00_2 \text{ (Input Mode)}
\end{equation}
\begin{equation}
\text{MODER}_{\text{PA4,5}} = 11_2 \text{ (Analog Mode)}
\end{equation}

Register configuration sequence:

\begin{algorithm}[H]
\caption{Detailed GPIO Configuration}
\begin{algorithmic}[1]
\State $\text{RCC}\rightarrow\text{AHBENR} \gets \text{RCC}\rightarrow\text{AHBENR} \lor \text{RCC\_AHBENR\_GPIOAEN}$
\State $\text{GPIOA}\rightarrow\text{MODER} \gets \text{GPIOA}\rightarrow\text{MODER} \land \neg(\text{GPIO\_MODER\_MODER0})$
\State $\text{GPIOA}\rightarrow\text{MODER} \gets \text{GPIOA}\rightarrow\text{MODER} \land \neg(\text{GPIO\_MODER\_MODER1})$
\State $\text{GPIOA}\rightarrow\text{MODER} \gets \text{GPIOA}\rightarrow\text{MODER} \land \neg(\text{GPIO\_MODER\_MODER2})$
\State $\text{GPIOA}\rightarrow\text{MODER} \gets \text{GPIOA}\rightarrow\text{MODER} \lor \text{GPIO\_MODER\_MODER4}$
\State $\text{GPIOA}\rightarrow\text{MODER} \gets \text{GPIOA}\rightarrow\text{MODER} \lor \text{GPIO\_MODER\_MODER5}$
\State $\text{GPIOA}\rightarrow\text{PUPDR} \gets \text{GPIOA}\rightarrow\text{PUPDR} \land \neg(\text{GPIO\_PUPDR\_PUPDR1} \lor \text{GPIO\_PUPDR\_PUPDR2})$
\end{algorithmic}
\end{algorithm}

\subsection{Timer Configuration}
\subsubsection{TIM2 Setup Parameters}

The timer configuration implements a free-running counter with the following characteristics:

\begin{equation}
T_{\text{count}} = \frac{1}{f_{\text{SYSCLK}}} = \frac{1}{48\text{MHz}} = 20.83\text{ns}
\end{equation}

\begin{equation}
\text{Maximum Period} = 2^{32} \times T_{\text{count}} = 89.48\text{s}
\end{equation}

Timer register configuration:

\begin{algorithm}[H]
\caption{TIM2 Initialization Sequence}
\begin{algorithmic}[1]
\State $\text{RCC}\rightarrow\text{APB1ENR} \gets \text{RCC}\rightarrow\text{APB1ENR} \lor \text{RCC\_APB1ENR\_TIM2EN}$
\State $\text{TIM2}\rightarrow\text{CR1} \gets 0x008C$ \Comment{Auto-reload, up-count, overflow stop}
\State $\text{TIM2}\rightarrow\text{PSC} \gets 0$ \Comment{No prescaling}
\State $\text{TIM2}\rightarrow\text{ARR} \gets 0xFFFFFFFF$ \Comment{Maximum period}
\State $\text{TIM2}\rightarrow\text{EGR} \gets 0x0001$ \Comment{Update generation}
\State $\text{NVIC\_SetPriority(TIM2\_IRQn, 0)}$ \Comment{Highest priority}
\end{algorithmic}
\end{algorithm}

\section{Analog Systems}
\subsection{ADC Implementation}
\subsubsection{ADC Calibration Process}

The ADC calibration sequence ensures accurate voltage measurements:

\begin{equation}
\text{ADC}_{\text{error}} = \text{ADC}_{\text{measured}} - \text{ADC}_{\text{actual}}
\end{equation}

Calibration algorithm:

\begin{algorithm}[H]
\caption{ADC Calibration Sequence}
\begin{algorithmic}[1]
\State $\text{ADC1}\rightarrow\text{CR} \gets \text{ADC\_CR\_ADCAL}$
\While{$\text{ADC1}\rightarrow\text{CR} = \text{ADC\_CR\_ADCAL}$}
    \State \text{Wait for calibration completion}
\EndWhile
\State $\text{ADC1}\rightarrow\text{SMPR} \gets 0x7$ \Comment{Maximum sampling time}
\State $\text{ADC1}\rightarrow\text{CHSELR} \gets \text{ADC\_CHSELR\_CHSEL5}$
\end{algorithmic}
\end{algorithm}

\subsubsection{ADC Sampling Process}

The ADC sampling follows:

\begin{equation}
V_{\text{measured}} = \frac{\text{ADC}_{\text{value}}}{4095} \times V_{\text{ref}}
\end{equation}

\begin{equation}
R_{\text{measured}} = \frac{\text{ADC}_{\text{value}}}{4095} \times 5000\Omega
\end{equation}

\subsection{DAC Implementation}

The DAC output voltage is calculated as:

\begin{equation}
V_{\text{out}} = \frac{\text{DAC}_{\text{value}}}{4095} \times V_{\text{ref}}
\end{equation}

DAC initialization sequence:

\begin{algorithm}[H]
\caption{DAC Initialization}
\begin{algorithmic}[1]
\State $\text{RCC}\rightarrow\text{APB1ENR} \gets \text{RCC}\rightarrow\text{APB1ENR} \lor \text{RCC\_APB1ENR\_DACEN}$
\State $\text{DAC}\rightarrow\text{CR} \gets \text{DAC}\rightarrow\text{CR} \land \neg(0x7)$
\State $\text{DAC}\rightarrow\text{CR} \gets \text{DAC}\rightarrow\text{CR} \lor \text{DAC\_CR\_EN1}$
\end{algorithmic}
\end{algorithm}

\section{Frequency Measurement System}
\subsection{Edge Detection and Timing}

The frequency measurement process follows:

\begin{equation}
f_{\text{measured}} = \frac{f_{\text{SYSCLK}}}{\text{TIM2\_CNT}}
\end{equation}

\begin{equation}
T_{\text{measured}} = \frac{\text{TIM2\_CNT}}{f_{\text{SYSCLK}}}
\end{equation}

Edge detection algorithm:

\begin{algorithm}[H]
\caption{Frequency Measurement Process}
\begin{algorithmic}[1]
\If{$(\text{EXTI}\rightarrow\text{PR} \land \text{register\_mask}) \neq 0$}
    \If{$(\text{TIM2}\rightarrow\text{CR1} \land \text{TIM\_CR1\_CEN}) = 0$}
        \State $\text{TIM2}\rightarrow\text{CNT} \gets 0$
        \State $\text{TIM2}\rightarrow\text{CR1} \gets \text{TIM2}\rightarrow\text{CR1} \lor \text{TIM\_CR1\_CEN}$
    \Else
        \State $\text{TIM2}\rightarrow\text{CR1} \gets \text{TIM2}\rightarrow\text{CR1} \land \neg(\text{TIM\_CR1\_CEN})$
        \State $\text{count} \gets \text{TIM2}\rightarrow\text{CNT}$
        \State $\text{period} \gets \frac{\text{count}}{\text{SystemCoreClock}}$
        \State $\text{frequency} \gets \frac{1}{\text{period}}$
    \EndIf
    \State $\text{EXTI}\rightarrow\text{PR} \gets \text{register\_mask}$
\EndIf
\end{algorithmic}
\end{algorithm}

\section{Mode Switching System}
\subsection{Button Debouncing and Mode Control}

The mode switching implements the following state machine:

\begin{equation}
\text{State}_{\text{next}} = \begin{cases}
\text{NE555}, & \text{if } \text{State}_{\text{current}} = \text{FGen} \land \text{Button} = 1 \\
\text{FGen}, & \text{if } \text{State}_{\text{current}} = \text{NE555} \land \text{Button} = 1 \\
\text{State}_{\text{current}}, & \text{otherwise}
\end{cases}
\end{equation}

Mode switching algorithm:

\begin{algorithm}[H]
\caption{Mode Switching Process}
\begin{algorithmic}[1]
\If{$(\text{EXTI}\rightarrow\text{PR} \land \text{EXTI\_PR\_PR0}) \neq 0$}
    \If{$(\text{GPIOA}\rightarrow\text{IDR} \land \text{GPIO\_IDR\_0}) \neq 0$}
        \While{$(\text{GPIOA}\rightarrow\text{IDR} \land \text{GPIO\_IDR\_0}) \neq 0$}
            \State \text{Wait for button release}
        \EndWhile
        \State $\text{funcGen\_mode} \gets \neg\text{funcGen\_mode}$
        \If{$\neg\text{funcGen\_mode}$}
            \State $\text{EXTI}\rightarrow\text{IMR} \gets \text{EXTI}\rightarrow\text{IMR} \land \neg(\text{EXTI\_IMR\_IM2})$
            \State $\text{EXTI}\rightarrow\text{IMR} \gets \text{EXTI}\rightarrow\text{IMR} \lor \text{EXTI\_IMR\_IM1}$
        \Else
            \State $\text{EXTI}\rightarrow\text{IMR} \gets \text{EXTI}\rightarrow\text{IMR} \land \neg(\text{EXTI\_IMR\_IM1})$
            \State $\text{EXTI}\rightarrow\text{IMR} \gets \text{EXTI}\rightarrow\text{IMR} \lor \text{EXTI\_IMR\_IM2}$
        \EndIf
    \EndIf
    \State $\text{EXTI}\rightarrow\text{PR} \gets \text{EXTI\_PR\_PR0}$
\EndIf
\end{algorithmic}
\end{algorithm}

\section{System Performance Analysis}
\subsection{Timing Characteristics}

The system timing characteristics are defined by:

\begin{equation}
T_{\text{measurement}} = \frac{1}{f_{\text{input}}}
\end{equation}

\begin{equation}
\text{Resolution} = \frac{1}{f_{\text{SYSCLK}}} = 20.83\text{ns}
\end{equation}

\begin{equation}
f_{\text{max}} = \frac{f_{\text{SYSCLK}}}{2} = 24\text{MHz}
\end{equation}

\subsection{Accuracy Analysis}

The measurement accuracy is affected by several factors:

\begin{equation}
\text{Error}_{\text{total}} = \sqrt{\text{Error}_{\text{quantization}}^2 + \text{Error}_{\text{clock}}^2 + \text{Error}_{\text{trigger}}^2}
\end{equation}

Where:
\begin{equation}
\text{Error}_{\text{quantization}} = \pm\frac{1}{2} \times \frac{1}{f_{\text{SYSCLK}}}
\end{equation}

\begin{equation}
\text{Error}_{\text{clock}} = \pm\text{PPM}_{\text{clock}} \times T_{\text{measurement}}
\end{equation}

\section{Interrupt Priority and Management}
\subsection{Priority Hierarchy}

The interrupt priority system follows:

\begin{equation}
\text{Priority}_{\text{effective}} = \begin{cases}
0, & \text{for TIM2\_IRQn} \\
0, & \text{for EXTI0\_1\_IRQn} \\
1, & \text{for EXTI2\_3\_IRQn}
\end{cases}
\end{equation}

\end{document}